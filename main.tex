\documentclass[12pt,a4paper]{article}

% 中文支持
\usepackage[UTF8]{ctex}

% 页面设置
\usepackage[top=2.5cm,bottom=2.5cm,left=2.5cm,right=2.5cm]{geometry}

% 数学公式
\usepackage{amsmath,amssymb,amsthm}

% 图表支持
\usepackage{graphicx}
\usepackage{float}

% 其他必要包
\usepackage{enumerate}
\usepackage{titlesec}
\usepackage{setspace}

% 字体设置 - 使用ctex的字体命令
\setCJKfamilyfont{fs}{FangSong}  % 仿宋
\newcommand{\fs}{\CJKfamily{fs}}

% 1.5倍行距
\onehalfspacing

% 标题格式设置
% 一级标题:三号黑体,顶格
\titleformat{\section}
{\heiti\fontsize{16pt}{24pt}\selectfont}
{\chinese{section}、}
{0em}
{}

% 二级标题:小三黑体,缩进2格
\titleformat{\subsection}
{\heiti\fontsize{15pt}{22.5pt}\selectfont}
{\hspace{2em}\arabic{subsection}.}
{0em}
{}

% 三级标题:小三黑体,缩进2格
\titleformat{\subsubsection}
{\heiti\fontsize{15pt}{22.5pt}\selectfont}
{\hspace{2em}\arabic{section}.\arabic{subsection}}
{0em}
{}

% 公式编号格式 (x.y)
\numberwithin{equation}{section}
\renewcommand{\theequation}{(\arabic{section}.\arabic{equation})}

% 正文字体:四号宋体
\renewcommand{\normalsize}{\fontsize{12pt}{18pt}\selectfont}

\begin{document}

% 标题部分
\begin{center}
{\heiti\fontsize{16pt}{24pt}\selectfont 赛题二}
\end{center}

% 作者信息(四号仿宋)
\begin{center}
{\fs\fontsize{12pt}{18pt}\selectfont 作者1 作者2 作者3 … 老师1(指导老师)}
\end{center}

% 学校和邮箱(小四宋体)
\begin{center}
{\songti\fontsize{10.5pt}{15.75pt}\selectfont 暨南大学 \; 邮箱}
\end{center}

\vspace{1em}

% 摘要部分
\noindent{\heiti\fontsize{12pt}{18pt}\selectfont 摘要:}{\songti\fontsize{12pt}{18pt}\selectfont 在此处输入摘要内容,约300~500字。应说明工作的目的、研究方法、结果和最终结论。要突出本论文的创造性成果或新的见解,语言力求精炼。为便于文献检索,应在本页下方另起一行注明本文的关键词(3~5个)。}

\vspace{1em}

\noindent{\heiti\fontsize{12pt}{18pt}\selectfont 关键词:}{\songti\fontsize{12pt}{18pt}\selectfont 关键词1;关键词2;关键词3;关键词4;关键词5}

\vspace{2em}

% 引言
{\centering\heiti\fontsize{16pt}{24pt}\selectfont 引言\par}
\vspace{1em}
在论文正文前,应简要阐述对赛题的分析、解题使用的主要方法和解题结果等内容。解题思路为将通过构造格将RLWE转化为CVP问题,再通过Kannan embedding将问题转化为SVP问题后使用Seiving求解。

% 正文开始
\section{求解环R中主理想$a(x)=R_q$的概率}

\subsection{概率计算方法}

这里是二级标题下的正文内容。所有正文使用四号宋体。

\subsubsection{具体实现步骤}



\subsection{结果}

\begin{equation}
p = \left(1 - \frac{1}{q^2}\right)^{128} \tag{1.1}
\end{equation}。


\section{题目二}

\subsection{题目2}

\subsection{算法优化}

\begin{equation}
\nabla f(x) = 2ax + b
\end{equation}

这个公式的编号为(2.2)。

\section{实验结果与分析}

\subsection{实验设置}

在这里描述实验的具体设置。

\subsection{结果分析}

在这里分析实验结果。

\section{结论}

总结论文的主要贡献和结论。

\end{document}